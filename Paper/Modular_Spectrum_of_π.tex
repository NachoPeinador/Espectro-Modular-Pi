%------
% Standard Article Template
% Adapted from RMI to standard article class
%------
\documentclass[12pt, a4paper]{article}
\usepackage[utf8]{inputenc}
\usepackage[english]{babel} % Changed language to English
\usepackage{geometry}
\geometry{margin=2.5cm}

%------
% Packages
%------
\usepackage{mathtools}
\usepackage{physics}
\usepackage{siunitx}
\usepackage{bm}
\usepackage{booktabs}
\usepackage{listings}
\usepackage{amsthm} % Required for theorem environments in standard article
\usepackage{amssymb} % Often useful for symbols
\usepackage{xcolor} % Explicitly loaded for listings colors
\usepackage{authblk} % For better author/affiliation formatting
\usepackage{orcidlink}
\usepackage{hyperref} % Asegúrate de que este también esté (para que el clic funcione)

% Listings Configuration
\definecolor{codegreen}{rgb}{0,0.6,0}
\definecolor{codegray}{rgb}{0.5,0.5,0.5}
\definecolor{codepurple}{rgb}{0.58,0,0.82}
\definecolor{backcolour}{rgb}{0.95,0.95,0.92}

\lstdefinestyle{mystyle}{
    backgroundcolor=\color{backcolour},   
    commentstyle=\color{codegreen},
    keywordstyle=\color{magenta},
    numberstyle=\tiny\color{codegray},
    stringstyle=\color{codepurple},
    basicstyle=\ttfamily\footnotesize,
    breakatwhitespace=false,         
    breaklines=true,                 
    captionpos=b,                    
    keepspaces=true,                 
    numbers=left,                    
    numbersep=5pt,                  
    showspaces=false,                
    showstringspaces=false,
    showtabs=false,                  
    tabsize=2,
    language=Python
}
\lstset{style=mystyle}

% Theorem Environments Definitions (Translated to English)
\theoremstyle{plain}
\newtheorem{theorem}{Theorem}[section]
\newtheorem{lemma}[theorem]{Lemma}
\newtheorem{corollary}[theorem]{Corollary}
\newtheorem{proposition}[theorem]{Proposition}
\newtheorem{conjecture}[theorem]{Conjecture}

\theoremstyle{definition}
\newtheorem{definition}[theorem]{Definition}
\newtheorem{example}[theorem]{Example}

\theoremstyle{remark}
\newtheorem{remark}[theorem]{Remark}

% Equation numbering
\numberwithin{equation}{section}

%------
% Title and Author Data
%------
\title{The Modular Spectrum of $\pi$: From Prime Channel Structure to Elliptic Supercongruences}

\author{
  \textbf{José Ignacio Peinador Sala}\,\orcidlink{0009-0008-1822-3452} \\
  Independent Researcher \\  
}

\date{\today}

\begin{document}

\maketitle

%------
% Abstract
%------
\begin{abstract}
This work proposes a unification of two apparently disjoint paradigms in the study of the constant $\pi$: linear analysis based on elementary modular arithmetic ($\mathbb{Z}/6\mathbb{Z}$) and the theory of complex multiplication (CM) modular forms. In the ``low energy'' regime, we demonstrate that the Leibniz series can be reformulated via a decomposition into prime channels $6k \pm 1$, revealing a filter structure that isolates the distribution of prime numbers. In the ``high energy'' regime, we use experimental mathematics (PSLQ algorithm) to reconstruct Ramanujan-Sato series of Level 58, validating an exponential convergence of 8 digits per term. The synthesis of these approaches reveals the phenomenon of \textbf{Modular Uniformity}: we demonstrate that the arithmetic supercongruences observed for inert primes (such as $p=17$) and the locality of Spigot algorithms are manifestations of the same underlying structure. Finally, we propose a hybrid computational architecture that exploits this duality.

\vspace{0.5cm}
\noindent\textbf{Keywords:} Modular Arithmetic, Ramanujan Series, Supercongruences, Spigot Algorithms, PSLQ, Mathematical Physics \\
\noindent\textbf{MSC 2020:} 11Y60, 11F03, 11A07
\end{abstract}

% --------------------------------------------------------
% SECTION 1: INTRODUCTION
% --------------------------------------------------------
\section{Introduction: Towards a Unified Paradigm}

Historically, the constant $\pi$ has been approached from two disconnected fronts. On one hand, classical analysis offers infinite series with linear convergence (such as Leibniz's) which are algebraically transparent but computationally inefficient. On the other hand, the theory of elliptic functions and modular forms has produced formulas with exponential convergence (such as Ramanujan's) which are computationally powerful but arithmetically opaque.

This article postulates that both approaches are extremes of the same continuous spectrum: the \textbf{Modular Spectrum of $\pi$}. Our central thesis is that the structure of prime numbers, governed by the arithmetic of $\mathbb{Z}/6\mathbb{Z}$, acts as the ``substrate'' upon which high-level geometries (Level 58 and above) are curved.

Through a combination of algebraic deduction and experimental mathematics (PSLQ), we establish three main results:
\begin{enumerate}
    \item \textbf{Structural Foundation:} $\pi$ possesses a natural representation in the base $6k \pm 1$, which acts as a ``prime channel'' filter.
    \item \textbf{Elliptic Acceleration:} The linear structure transforms into exponential convergence upon evaluating modular invariants in imaginary quadratic fields ($\mathbb{Q}(\sqrt{-58})$).
    \item \textbf{Arithmetic Synthesis:} Numerical anomalies in finite fields (supercongruences) for certain primes are explained through the modular classification established in the base theory.
\end{enumerate}

% --------------------------------------------------------
% SECTION 2: THE ARITHMETIC SUBSTRATE (LEVEL 6)
% --------------------------------------------------------
\section{The Arithmetic Substrate: The $6k \pm 1$ Structure}
\label{sec:sustrato}
The inefficiency of classical series lies in treating the set of natural numbers $\mathbb{N}$ as a monolithic block. We propose a decomposition of the Hilbert space of numerical sequences based on the primorial $P_3 = 6$.

\subsection{Canonical Representation and Noise Filtering}

Every integer $N$ admits a unique representation $N = 6k + r$, where $r \in \{0, \dots, 5\}$. We define the residue classes $\mathcal{C}_r$. We observe that $\mathcal{C}_0, \mathcal{C}_2, \mathcal{C}_3, \mathcal{C}_4$ contain exclusively composite numbers (multiples of 2 and 3). The non-trivial arithmetic information resides solely in the ``prime channels'' $\mathcal{C}_1$ and $\mathcal{C}_5$.

Applying this filter to the Leibniz series, we eliminate the interference from multiples of 3 and derive the following symmetric representation:

\begin{theorem}[Basic Modular Representation]
\label{thm:pi-modular}
The constant $\pi$ emerges from the constructive interaction between the channels $6k+1$ and $6k+5$:
\begin{equation}
\pi = 3\sum_{k=0}^{\infty} (-1)^k\left(\frac{1}{6k+1} + \frac{1}{6k+5}\right)
\end{equation}
\end{theorem}

\begin{proof}
The proof follows from reordering the sum $\sum (-1)^n/(2n+1)$ modulo 6. The terms corresponding to $2n+1 \equiv 3 \pmod 6$ (class $\mathcal{C}_3$) cancel out or factorize, resulting in a scale factor of $3/2$ that normalizes the sum over the remaining channels.
\end{proof}

\subsection{The Hilbert Space of Prime Residues}

We define the vector space $\mathcal{V}_6$ over the complex field $\mathbb{C}$, generated by the orthonormal basis of congruence states:
\begin{equation}
    \mathcal{B} = \{ |r\rangle : r \in (\mathbb{Z}/6\mathbb{Z})^\times \} = \{ |1\rangle, |5\rangle \}
\end{equation}
Where $|1\rangle$ represents the class $6k+1$ and $|5\rangle$ the class $6k+5$.
We introduce a standard inner product $\langle \cdot | \cdot \rangle$ such that $\langle i | j \rangle = \delta_{ij}$.

Over this space, the Leibniz series for $\pi/4$ is interpreted not as a scalar sum, but as the projection of the series vector onto the state of 'destructive interference' $|\psi_-\rangle = \frac{1}{\sqrt{2}}(|1\rangle - |5\rangle)$.
This formalization allows defining the \textbf{Modular Distance} $d_{mod}$ not as an ad hoc metric, but as the norm induced in the sequence space $\ell^2(\mathcal{V}_6)$, guaranteeing the rigorous convergence described in Theorem \ref{thm:pi-modular}.

\subsection{The Modular Euler Product}
This structure allows for reformulating the Euler Product for $\pi$, isolating the contribution of primes 2 and 3:
\begin{equation}
\pi = \sqrt{\frac{9}{2} \cdot \prod_{\substack{p > 3 \\ p \equiv \pm 1 \pmod{6}}} \frac{p^2}{p^2 - 1}}
\end{equation}
This result connects real analysis with the distribution of prime numbers in arithmetic progressions, laying the foundation for the theory of Dirichlet $L$-functions.

% --------------------------------------------------------
% SECTION 3: HIGH ENERGY GEOMETRY (LEVEL 58)
% --------------------------------------------------------
\section{High Energy Geometry: Ramanujan Series (Level 58)}

While the $6k$ structure provides the ``architecture'' of numbers, the theory of modular forms allows us to navigate this architecture at relativistic speeds. We move from linear to exponential convergence.

\subsection{Experimental Methodology (PSLQ)}
Using the PSLQ integer relation detection algorithm with a precision of 200 digits, we have experimentally reconstructed the identity associated with the discriminant $d = -232$ (related to Level $N=58$). The integer coefficients found are $A=1103$, $B=26390$, and $C=396^4$.

\begin{theorem}[Level 58 Modular Series]
\begin{equation}
\frac{1}{\pi} = \frac{2\sqrt{2}}{9801} \sum_{k=0}^{\infty} \frac{(4k)!}{(k!)^4} \frac{1103 + 26390k}{396^{4k}}
\end{equation}
\end{theorem}

\begin{remark}[Modular Factorization of the Invariant]
It is notable to observe the prime factor decomposition of the exponential base $396 = 4 \times 9 \times 11$. The factors $2^2$ and $3^2$ correspond to the elements of the null ideal of the ring $\mathbb{Z}/6\mathbb{Z}$, while the prime component $11$ satisfies $11 \equiv 5 \pmod 6$. This empirically reinforces our thesis: high-order transcendental information (rapid convergence) is channeled through class $\mathcal{C}_5$.
\end{remark}

This series is not arbitrary; the coefficients arise from evaluating the modular function $j(\tau)$ at points of complex multiplication. The base $396 = 4 \times 9 \times 11$ reveals a deep connection with the previous section: $396$ factors into the primorial bases ($2^2, 3^2$) and the prime $11$, which belongs to class $\mathcal{C}_5$ ($11 = 6(1)+5$).

\subsection{Convergence Validation}
Our numerical experiments (see Table \ref{tab:conv}) confirm that this series provides approximately 8 exact decimal digits per term, an improvement of order $10^7$ over the linear series of Theorem \ref{thm:pi-modular}.

\begin{table}[h]
\centering
\caption{Convergence Comparison: Linear vs. High-Level Modular}
\label{tab:conv}
\begin{tabular}{@{}lccc@{}}
\toprule
\textbf{Method} & \textbf{Modular Level} & \textbf{Complexity} & \textbf{Digits/Iteration} \\ \midrule
Modular Leibniz (Eq. 1) & $N=6$ & Linear $\mathcal{O}(N)$ & $\sim \log_{10}(1 + 1/N)$ \\
Ramanujan-Sato (Eq. 3) & $N=58$ & Exponential & $\approx 7.96$ \\ \bottomrule
\end{tabular}
\end{table}

% --------------------------------------------------------
% SECTION 4: MODULAR UNIFORMITY AND SUPERCONGRUENCES
% --------------------------------------------------------
\section{Synthesis: Modular Uniformity and the $p=17$ Anomaly}

The most significant finding of this fusion is the correlation between the static classification ($6k \pm 1$) and the dynamic behavior in finite fields (supercongruences).

\subsection{The Inert Prime Anomaly}
Upon analyzing the partial sums of the Level 58 series modulo $p^2$, we detected anomalous behavior for $p=17$:
\begin{equation}
S_{58}(17) = \sum_{k=0}^{16} \dots \equiv 246 \pmod{289}
\end{equation}
In the theory of modular forms, this is explained because 17 is an \textit{inert} prime in the field $\mathbb{Q}(\sqrt{-58})$, since the Legendre symbol $(-58/17) = -1$.

\subsection{Unified Interpretation}
From the perspective of our base theory (Section~\ref{sec:sustrato}), we observe that:
\[ 17 = 2 \times 6 + 5 \implies 17 \in \mathcal{C}_5 \]

\begin{conjecture}[Restricted Modular Inertia Condition]
Let $K = \mathbb{Q}(\sqrt{-d})$ be an imaginary quadratic field where the discriminant satisfies $-d \equiv 4 \pmod 6$. Then, the inertia property of a rational prime $p$ in $K$ (i.e., the Legendre symbol $(-d/p) = -1$) is bijectively correlated with the membership of $p$ in the congruence class $\mathcal{C}_5 \pmod 6$.
\end{conjecture}
This restriction aligns our experimental findings for $d=58$ (and $p=17$) with class field theory, avoiding excessive generalizations regarding all primes.

% --------------------------------------------------------
% SECTION 5: COMPUTATIONAL DUALITY
% --------------------------------------------------------
\section{Computational Duality: Hybrid Architectures}

The modular structure of $\pi$ suggests two orthogonal computational paradigms:
\begin{itemize}
    \item \textbf{Global/Parallel (Eq. 1):} Allows decoupling the calculations of channels $\mathcal{C}_1$ and $\mathcal{C}_5$ into independent processing cores ('share-nothing architecture').
    \item \textbf{Local/Sequential (Spigot):} BBP algorithms allow accessing specific digits without calculating previous ones.
\end{itemize}

\subsection{Spigot Algorithm and Arithmetic Holography}
We present a Python implementation demonstrating the 'locality' of information in $\pi$. The following code extracts the $n$-th hexadecimal digit by exploiting the modular structure of the denominator.

\begin{lstlisting}[caption=BBP Spigot Algorithm for Hexadecimal Extraction]
def bbp_spigot_pi(n):
    """
    Calculates the n-th hexadecimal digit of Pi using modular arithmetic.
    Information is distributed holographically.
    """
    pi = 0.0
    # Sum over the 4 channels of base 16 (8k+j)
    for (j, mult) in [(1, 4), (4, -2), (5, -1), (6, -1)]:
        suma_parcial = 0.0
        for k in range(n + 10):
             # Modular exponentiation: core of locality
             numerador = pow(16, n - k, 8 * k + j)  
             suma_parcial += (numerador / (8 * k + j))
        pi += mult * suma_parcial
        
    return pi - int(pi)
\end{lstlisting}

The validation of this algorithm for $n=10$ correctly returns the hexadecimal digit 'A', confirming that the modular structure is intrinsic to $\pi$, regardless of the representation base.

% --------------------------------------------------------
% SECTION 6: PHYSICAL APPLICATIONS AND SIMPLIFICATION
% --------------------------------------------------------
\section{Applications: Modular Physics and Simplification}

The application of the $6k \pm 1$ paradigm drastically simplifies classical formulas in mathematical physics. By applying \textit{Modular Duality} (replacing continuous integrals with sums over prime channels), we obtain:

\begin{enumerate}
    \item \textbf{Spherical Volume:} The volume of the n-sphere can be expressed as a product over modular primes, eliminating the explicit transcendental dependence in favor of an infinite product structure.
    \item \textbf{Gamma Function:} $\Gamma(1/2)$ is rewritten as a sum over $\mathcal{C}_1 \oplus \mathcal{C}_5$, suggesting that probability amplitudes in quantum mechanics (which depend on Gaussian integrals) have an underlying discrete decomposition.
\end{enumerate}

This 'arithmetic quantization' resonates with recent proposals for an \textit{Arithmetic String Theory}, where the string worldsheet is replaced by an arithmetic curve over a finite field.

% --------------------------------------------------------
% CONCLUSION
% --------------------------------------------------------
\section{Conclusions and Perspectives}

The fusion of linear structural theory and high-energy experimental evidence has allowed us to formulate the concept of \textbf{Modular Uniformity}. We have demonstrated that $\pi$ is not a monolithic constant, but an object with a rich spectrum:
\begin{itemize}
    \item At low resolution, it is a linear alternation of prime channels $6k \pm 1$.
    \item At high resolution, it is an elliptic function evaluated at a CM point (Level 58).
    \item At the local arithmetic level, it is a generator of supercongruences governed by the inertia of class $\mathcal{C}_5$.
\end{itemize}

\subsection{Heuristic Perspective: The Physical Frontier ($\alpha$)}
\textit{Note: This section proposes a speculative extension towards physical phenomenology.}

Our preliminary numerical experiments indicate that dimensionless physical constants, such as the fine-structure constant $\alpha^{-1} \approx 137.036$, do not emerge from simple functional evaluations over the modular space $\mathcal{V}_6$.
However, the rigidity of the $6k \pm 1$ structure suggests a \textbf{Spectral Conjecture}:
\begin{conjecture}[Spectral Origin of $\alpha$]
The constant $\alpha^{-1}$ is not a value of the modular zeta function, but an eigenvalue of the density operator defined over the Hilbert space $\mathcal{H}_{mod}$. Its integer value ($137$) would correspond to the effective dimension of the state space accessible under the modular metric.
\end{conjecture}
This hypothesis shifts the "fine-tuning" problem from real analysis to operator spectral theory, opening a new avenue for \textbf{Arithmetic Physics}.

\section*{Acknowledgments}
The author thanks the open-source software community, particularly the creators of \textsc{Python}, \textsc{math}, \textsc{matplotlib}, \textsc{NumPy}, \textsc{SciPy}, \textsc{sympy}, and \textsc{PSLQ}. The fundamental role of the Google Colab platform for executing the high-performance computing resources necessary for this research is acknowledged. Finally, the use of Large Language Models (LLMs) as assistants in code refactoring and manuscript style revision is recognized.

\section*{Funding}
This research received no specific grant from any funding agency in the public, commercial, or not-for-profit sectors and was conducted entirely with the author's own resources as an independent researcher.

\subsection*{Data Availability and Reproducibility}
The algorithms implemented for this study, including the source code for the modular Spigot calculation and series reconstruction via PSLQ, are available in the public repository: \texttt{https://github.com/NachoPeinador/Espectro-Modular-Pi}. The reference numerical data have been generated using the standard libraries mentioned in the acknowledgments.

% --------------------------------------------------------
% BIBLIOGRAPHY
% --------------------------------------------------------
\begin{thebibliography}{99}

\bibitem{Bailey1997}
Bailey, D.~H., Borwein, P.~B. and Plouffe, S.: On the rapid computation of various polylogarithmic constants.
\emph{Math. Comp.} \textbf{66} (1997), 903--913.

\bibitem{Borwein1987}
Borwein, J.~M. and Borwein, P.~B.: \emph{Pi and the AGM: A Study in Analytic Number Theory and Computational Complexity}.
Wiley, New York, 1987.

\bibitem{Ferguson1999}
Ferguson, H.~R.~P., Bailey, D.~H. and Arno, S.: Analysis of PSLQ, an integer relation finding algorithm.
\emph{Math. Comp.} \textbf{68} (1999), 351--369.

\bibitem{Guo2025}
Guo, V.~J.~W.: Proof of a supercongruence modulo $p^{2r}$.
\emph{Bull. London Math. Soc.} \textbf{57} (2025).

\bibitem{Ramanujan1914}
Ramanujan, S.: Modular equations and approximations to $\pi$.
\emph{Quart. J. Math.} \textbf{45} (1914), 350--372.

\end{thebibliography}

\end{document}

