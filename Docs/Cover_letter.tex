\documentclass[11pt]{letter}
\usepackage[utf8]{inputenc}
\usepackage[spanish]{babel}
\usepackage{mathptmx}
\usepackage[margin=1in]{geometry}
\usepackage{hyperref}

\address{
José Ignacio Peinador Sala \\
Calle Florencia 1 \\
47007 Valladolid, España \\
Email: joseignacio.peinador@gmail.com
}

\signature{
José Ignacio Peinador Sala \\
Investigador Independiente
}

\begin{document}

%\begin{letter}{
Editor Jefe \\
Revista Matemática Iberoamericana \\
%\end{letter}

\opening{Estimado Editor Jefe:}

Tengo el honor de someter a consideración de la \textbf{Revista Matemática Iberoamericana} nuestro manuscrito titulado:

\textbf{``El Espectro Modular de $\pi$: De la Estructura de Canales Primos a las Supercongruencias Elípticas''}

Este trabajo representa una contribución fundamental hacia la unificación de dos paradigmas aparentemente disjuntos en la teoría de números: el análisis aritmético elemental basado en $\mathbb{Z}/6\mathbb{Z}$ y la teoría de formas modulares de multiplicación compleja.

\vspace{0.5cm}

\textbf{Contribuciones principales:}

\begin{itemize}
\item \textbf{Teoría Estructural:} Demostramos que $\pi$ posee una representación natural en la base $6k \pm 1$, actuando como un filtro de ``canales primos'' que aísla la distribución de números primos.

\item \textbf{Aceleración Modular:} Reconstruimos experimentalmente series de Ramanujan-Sato de Nivel 58 mediante el algoritmo PSLQ, validando una convergencia exponencial de 8 dígitos por término.

\item \textbf{Síntesis Unificadora:} Establecemos que las supercongruencias aritméticas para primos inertes (ej. $p=17$) y la localidad de algoritmos Spigot son manifestaciones de la misma estructura modular subyacente.

\item \textbf{Arquitectura Computacional:} Proponemos un paradigma híbrido que explota esta dualidad para el cálculo eficiente de constantes matemáticas.
\end{itemize}

\vspace{0.5cm}

Consideramos que este manuscrito es particularmente apropiado para la \textbf{Revista Matemática Iberoamericana} debido a:

\begin{itemize}
\item Su enfoque \textbf{interdisciplinario} que conecta teoría de números, análisis y computación
\item La combinación de \textbf{rigor teórico} con \textbf{matemática experimental} de alta precisión
\item El potencial para abrir nuevas líneas de investigación en \textbf{física aritmética} y \textbf{teoría espectral modular}
\end{itemize}

\vspace{0.5cm}

Todos los códigos y datos están disponibles en nuestro repositorio público para garantizar la completa reproducibilidad de los resultados.

Agradecemos su consideración y quedamos a disposición para cualquier consulta o modificación que pudiera ser necesaria.

\closing{Atentamente,}

\encl{Manuscrito, Resumen Ejecutivo, enlace a repositorio con código Fuente:} \texttt{https://github.com/NachoPeinador/Espectro-Modular-Pi}

%\end{letter}
\end{document}
