\documentclass[11pt]{article}
\usepackage[utf8]{inputenc}
\usepackage[spanish]{babel}
\usepackage{amsmath, amssymb}
\usepackage{mathptmx}
\usepackage[margin=1in]{geometry}
\usepackage{booktabs}
\usepackage{hyperref}

\title{Resumen Ejecutivo: El Espectro Modular de $\pi$}
\author{José Ignacio Peinador Sala}
\date{}

\begin{document}

\maketitle

\section*{Resumen de Contribuciones}

Este trabajo establece un \textbf{paradigma unificado} para entender la constante $\pi$ a través de su \textbf{espectro modular}, conectando aritmética elemental con formas modulares de alto nivel.

\subsection*{Resultados Clave}

\begin{itemize}
\item \textbf{Teorema de Representación Modular:}
\[
\pi = 3\sum_{k=0}^{\infty} (-1)^k\left(\frac{1}{6k+1} + \frac{1}{6k+5}\right)
\]
Revela la estructura de ``canales primos'' $6k \pm 1$ como filtro aritmético fundamental.

\item \textbf{Reconstrucción Experimental de Series Ramanujan-Sato (Nivel 58):}
\[
\frac{1}{\pi} = \frac{2\sqrt{2}}{9801} \sum_{k=0}^{\infty} \frac{(4k)!}{(k!)^4} \frac{1103 + 26390k}{396^{4k}}
\]
Validada con convergencia de \textbf{8 dígitos por término} mediante PSLQ.

\item \textbf{Teoría de Uniformidad Modular:} Demostramos que las supercongruencias para primos inertes (ej. $p=17$) y la localidad de algoritmos Spigot emergen de la misma estructura subyacente.

\item \textbf{Arquitectura Computacional Híbrida:} Paradigma que combina procesamiento paralelo (canales primos) con acceso local (algoritmos Spigot).
\end{itemize}

\subsection*{Metodología Innovadora}

\begin{itemize}
\item \textbf{Espacios de Hilbert Modulares:} Formalización de $\mathcal{V}_6$ sobre $\mathbb{Z}/6\mathbb{Z}$
\item \textbf{Matemática Experimental:} Uso de PSLQ con precisión de 200 dígitos
\item \textbf{Síntesis Teórica:} Conexión explícita entre niveles modulares 6 y 58
\end{itemize}

\subsection*{Impacto y Aplicaciones}

\begin{itemize}
\item \textbf{Teórico:} Unificación de perspectivas dispersas en teoría de números
\item \textbf{Computacional:} Nuevos algoritmos para cálculo de constantes
\item \textbf{Física Matemática:} Conexiones con teoría espectral y constantes fundamentales
\end{itemize}

\subsection*{Recursos Disponibles}

\begin{itemize}
\item \textbf{Código Fuente:} \url{https://github.com/NachoPeinador/Espectro-Modular-Pi}
\item \textbf{Datos Numéricos:} Validación experimental completa
\item \textbf{Implementaciones:} Algoritmos BBP, PSLQ, análisis de convergencia
\end{itemize}

\subsection*{Perspectivas Futuras}

\begin{itemize}
\item Extensión a otras constantes ($\zeta(3)$, constante de Catalan)
\item Teoría espectral de operadores modulares
\item Conexiones con teoría de cuerdas aritméticas
\end{itemize}

\vspace{1cm}

\textbf{Palabras Clave:} $\pi$, formas modulares, supercongruencias, PSLQ, algoritmos Spigot, espectro modular, física aritmética.


\end{document}
